\newif\ifPAPER  
\PAPERtrue % select either slide or note
%\PAPERfalse  

\def\g4{{\sf Geant4}}

\newcommand{\codeAlgorithm}[1]{
\addcontentsline{toc}{section}{Résumé}
\begin{center}\fbox{\parbox{12cm}{\bf #1}}\end{center}}

\newcommand{\cppintro}[1]{
\lstset{language=C,
caption= #1 ,
label=listing:boundary}}

\def\cppstart{\begin{lstlisting}}
\def\cppend{\end{lstlisting}}

\newif\ifCITENOTE 
\CITENOTEtrue

\ifPAPER
\documentclass[twoside,floatfix,a4wide]{d}
\usepackage{multirow}
\usepackage{url}
\usepackage{listings}
\usepackage{graphicx}
\usepackage{wrapfig}
%\usepackage{makeidx}
\usepackage{subfig}
\usepackage{fancyhdr}
%\usepackage{asymptote}
\usepackage{amsmath}
\usepackage{verbatim} % for comment
\usepackage{eurosym} 
\usepackage{color} % for definecolor
\usepackage[colorlinks,bookmarks=true]{hyperref}

\numberwithin{equation}{section} % reguires amsmath-package
\pagestyle{fancy}
\fancyhead{} % clear all fields

\fancyhead[L]{\it {Helsinki, May 4, 2009}} % Left Odd, Right Even 
\fancyfoot[L]{G.Danielsen - Simulating carbon beam fragmentation on water phantom with the Geant4 INCL/ABLA models} 
\fancyhead[R]{\thepage}
\fancyfoot[C]{}
\newcommand{\urltilde}[1]{\texttt{#1}} % solves the tilde problem

\graphicspath{{images/}}
\DeclareGraphicsRule{.eps.gz}{eps}{.eps.bb}{`gunzip -c #1} % zipped images

\definecolor{light-gray}{gray}{0.95}
\definecolor{dark-gray}{gray}{0.30}
\definecolor{orange}{rgb}{1,0.5,0}
\definecolor{dark-blue}{cmyk}{1,0.5,0.5,0}
\hypersetup{
    bookmarks=true,         % show bookmarks bar?
    unicode=false,          % non-Latin characters in Acrobat’s bookmarks
    pdftoolbar=true,        % show Acrobat’s toolbar?
    pdfmenubar=true,        % show Acrobat’s menu?
    pdffitwindow=true,      % page fit to window when opened
    pdftitle={My title},    % title
    pdfauthor={Author},     % author
    pdfsubject={Subject},   % subject of the document
    pdfnewwindow=true,      % links in new window
    pdfkeywords={keywords}, % list of keywords
    colorlinks=true,        % false: boxed links; true: colored links
    linkcolor=dark-blue,          % color of internal links
    citecolor=dark-blue,        % color of links to bibliography
    filecolor=dark-blue,         % color of file links
    urlcolor=dark-blue            % color of external links
}
\begin{document}

\title{Simulating carbon beam fragmentation on water phantom with the Geant4 INCL/ABLA models}


\author{Gillis Danielsen$^1$ mentored by A.~Heikkinen$^2$} 
\affiliation{$^1$ Helsinki University of Technology}
\affiliation{$^2$ Helsinki Institute of Physics, P.O. Box 64, FIN-00014 University of Helsinki (Finland)}
\begin{titlepage}
\pagestyle{empty}
\begin{center}
rev. 001-2009\\
\vspace{7.5 cm}
\Huge
Simulating carbon hhhbeam fragmentation on water phantom with the Geant4 INCL/ABLA models\\

\vspace{5cm}

\Large
Gillis Danielsen, Bachelor's Thesis\\
Helsinki University of Technology\\

    \vspace{0,2cm}
  \end{center}

\end{titlepage}


\begin{abstract}
This work focuses on the simulation of carbon beams in a water phantom using GEANT4 code. Results will be compared to experimental data made available by the GSI Darmstadt/E.Haettner.

\footnote{Bachelor's thesis produced in the framework of the Finnish CERN Summer Training 2009.}
\end{abstract}
\maketitle
\thispagestyle{fancy}

\tableofcontents

\section{Introduction}
\begin{itemize}
\item CERN, CEA and HIP
\item Simulations
\item $C_{12}$ beams in water
\item reference data provided from GSI (E. Haettner, Master's thesis, KTH)
\item applications in Hadron treatment
\item Structure of this report
\end{itemize}
This paper focuses on the simulation of carbon beams in a water phantom using GEANT4 code as a medical application. GEANT4 is a multi-purpose physics simulation package developed at CERN, Switzerland. GEANT4 currently hosts multiple models suitable for the experiment, but this paper will focus on the INCL and ABLA models developed as a collaboration of scientists at CERN, HIP and CEA (tarkenna, nimet).

The history of radiation treatments dates back to the late 19th century when W.K. Roentgen discovered the X-rays. It was not long before these photon rays were being used to treat malign tissue. The first methods used we're on todays standards very crude, and much work has gone into perfecting the treatments in order to minimize the effect of the rays on the surrounding healthy tissue and maximize the dose delivered to the malign tissue.

However, due to the statistical nature of the photon interactions, a beam of many photons is
exponentially attenuated yielding an exponential decrease of the dose with the depth. To
obtain a higher dose in the tumor than in the surrounding normal tissue, many irradiation
fields are used. The cost of this method is that a large volume of the normal tissue
will suffer from a high dose. By replacing the x-rays with high energy photons, the dose
maximum is shifted a few centimeters deeper and the exponential decrease is more shallow,
which improves the ratio between dose in the tumors and in the normal tissue.

Although very sophisticated variations of the photon treatments have been developed, such as Intensity Modulated RadioTherapy (IMRT) photon therapies still produce considerable harm to the surrounding tissue. A much younger and less widely used technology is that of hadron-based radiotherapies. Hadrons are charged particles and therefore react to tissue in a much different way than photons. This is due to the fact that the primary halting force is electromagnetic interaction. Hadrons produce a much sharper energy-loss, a so-called bragg-peak firstdocumented by Bragg in ....-

An alternative to hadrons are carbon beams. The carbon beams produce a similar but considerably sharper Bragg-peak to the one of the protons. However, heavyer ions such as carbon will also produce smaller particles due to fragmentation, which yields a considerable ``tail``.

hadron and heavy ion treatments have been pioneered by berkley

The history of using photons in medical applications started at the end of the 19th century.
The world was excited about the new rays, the X-rays, discovered by Roentgen 1895. Soon
the x-rays were used for both imaging and treatment of malign tissue. Since then, the
photon therapy has become more and more sophisticated in order to increase the dose
given to the tumors and reduce the dose for the normal tissue.

Because of the statistical nature of the photon interactions, a beam of many photons is
exponentially attenuated yielding an exponential decrease of the dose with the depth. To
obtain a higher dose in the tumor than in the surrounding normal tissue, many irradiation
fields are used. The cost of this method is that a large volume of the normal tissue
will suffer from a high dose. By replacing the x-rays with high energy photons, the dose
maximum is shifted a few centimeters deeper and the exponential decrease is more shallow,
which improves the ratio between dose in the tumors and in the normal tissue.

Carbon beams are used in medical applications, due to the possibility of much more precise energy deposition in human tissue in comparison to the more conventional photon-treatments widely available. Photons have been used for medical treatments since the invention of the x-rays by W.K. Roentgen in 1895. Treatments involving hadrons and heavyer ions, however, are relatively new and in many cases still experimental.

The aim of this paper is therefore both to provide good reference data for the standardization work involved in hadron treatments as well as to evaluate the feasibility of the GEANT4 models for such simulations by comparison to experimental data.

Past papers related to carbon beam therapy include  works by E. Haettner~\cite{ehaettner} and K. Guntzer-Marx et. al.(2), which provide valuable experimental data for comparison.

Gunzter-Marx et. al. Paper focuses on a measurment on the effects of fragmentation particles to the tissue during hadron therapy. The paper presents an experiment where a carbon-beam is fired into a 12.78cm thick water phantom according to the experiment schematic [...]

The results are then plotted against the Monte Carlo codes PHITS and ATIMA, with a descent fit to the experimental data.

E Haettner's Master's thesis presents the measurement experiment 
\section{Theory}
\begin{equation}
 \frac{dE}{dx} = \frac{4 \pi}{m_e c^2} \cdot \frac{nz^2}{\beta^2} \cdot \left(\frac{e^2}{4\pi\varepsilon_0}\right)^2 \cdot \left[\ln \left(\frac{2m_e c^2 \beta^2}{I \cdot (1-\beta^2)}\right) - \beta^2\right]
\label{bethebloch}
\end{equation}


\begin{itemize}
 \item E. Haettner
 \item ABLA documentation
\end{itemize}
\begin{figure}
\begin{center}
\includegraphics[width=0.8\textwidth]{images/ablationabration.png}  
\caption{Schematic of the physics involved.}
 \label{fig:ablationabration}
 \end{center}
 \end{figure}
\begin{figure} 
\begin{center}
\includegraphics[width=0.8\textwidth]{images/inclScematic.png}  
\caption{\label{fig:inclschematic} Schematic diagram of INCL and ABLA models.}
 
 \end{center}
 \end{figure}


\section{Simulation setup}
\begin{itemize}
\item Experiment data from GSI
\item target description
\item Characterization of the beam
\item Simulation of detectors
\item Implementation details
\end{itemize}

\section{Results and comparison to GSI data}
\begin{itemize}
\item comparison (Haettner 5.4.1)
\item energy distribution of fragments
\end{itemize}

\section{Conclusion}
\begin{itemize}
\item Raflaava yhteenveto
\end{itemize}

\section{Appendices}
\begin{itemize}
\item code examples
\item runtime log
\end{itemize}





\bibliographystyle{plain} \bibliography{refs.bib} 
\end{document}

\else

\fi