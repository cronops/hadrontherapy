\documentclass[twoside,floatfix,a4wide]{d}

\usepackage{multirow}
\usepackage{url}
\usepackage{listings}
\usepackage{graphicx}
\usepackage{wrapfig}
\usepackage{makeidx}
\usepackage{subfig}
\usepackage{fancyhdr}
%\usepackage{asymptote}
\usepackage{amsmath}
\usepackage{verbatim} % for comment
\usepackage{eurosym} 
\usepackage{color} % for definecolor
\usepackage[colorlinks,bookmarks=true]{hyperref}
\usepackage{attachfile} 
\numberwithin{equation}{section} % reguires amsmath-package
\pagestyle{fancy}
\fancyhead{} % clear all fields

\fancyhead[L]{\it {Geneva, June 6, 2009}} % Left Odd, Right Even 
%\fancyfoot[L]{A.~Heikkinen: Report of Geant4 actictivity and outlook (June 2009)} 
\fancyhead[R]{\thepage}
\fancyfoot[C]{}
\newcommand{\urltilde}[1]{\texttt{#1}} % solves the tilde problem

\graphicspath{{images/}}
\DeclareGraphicsRule{.eps.gz}{eps}{.eps.bb}{`gunzip -c #1} % zipped images

\definecolor{light-gray}{gray}{0.95}
\definecolor{dark-gray}{gray}{0.30}
\definecolor{orange}{rgb}{1,0.5,0}
\definecolor{dark-blue}{cmyk}{1,0.5,0.5,0}
\hypersetup{
    bookmarks=true,         % show bookmarks bar?
    unicode=false,          % non-Latin characters in Acrobat’s bookmarks
    pdftoolbar=true,        % show Acrobat’s toolbar?
    pdfmenubar=true,        % show Acrobat’s menu?
    pdffitwindow=true,      % page fit to window when opened
    pdftitle={My title},    % title
    pdfauthor={Author},     % author
    pdfsubject={Subject},   % subject of the document
    pdfnewwindow=true,      % links in new window
    pdfkeywords={keywords}, % list of keywords
    colorlinks=true,        % false: boxed links; true: colored links
    linkcolor=dark-blue,          % color of internal links
    citecolor=dark-blue,        % color of links to bibliography
    filecolor=dark-blue,         % color of file links
    urlcolor=dark-blue            % color of external links
}
\begin{document}

\lstset{ % General settings
language=c++,                   % choose the language of the code
basicstyle=\ttfamily \tiny,     % the size of the fonts that are used for the code \footnotsize
numbers=left,                   % where to put the line-numbers
numberstyle=\small,             % the size of the fonts that are used for the line-numbers
stepnumber=2,                   % the step between two line-numbers. If it's 1 each line will be numbered
numbersep=10pt,                 % how far the line-numbers are from the code
showspaces=false,               % show spaces adding particular underscores
showstringspaces=false,         % underline spaces within strings
showtabs=false,                 % show tabs within strings adding particular underscores
frame=,                         % adds a frame around the code (single)
tabsize=2,                      % sets default tabsize to 2 spaces
captionpos=t,                   % sets the caption-position: top (t), bottom (b)
breaklines=true,                % sets automatic line breaking
breakatwhitespace=false,        % sets if automatic breaks should only happen at whitespace
escapeinside={\%*}{*)},         % if you want to add a comment within your code
caption=footnote, 
label=listing:relRef
}

%\begin{asydef}
%usepackage("bm");
%\end{asydef}

\title{Heavy charged-particle interaction data for radiotherapy -- 
      Report of the Geant4 actictivity and outlook (June 2009)
\footnote{2nd IAEA RCM on Heavy charged-particle interaction data for radiotherapy, 
                 8-12 June 2009 INFN-LNS Catania, Italy}}

\author{A.~Heikkinen} 
\affiliation{Helsinki Institute of Physics, P.O. Box 64, FIN-00014 University of Helsinki (Finland)}


\begin{abstract}
This report summarises Geant4 activity until June 2009, 
and outlines future plans in the Coordinated Research Project (CRP)
for {\em Heavy charged-particle interaction data for radiotherapy}.
\end{abstract}

\maketitle
\thispagestyle{fancy}

%\tableofcontents

\section{Geant4 collaborators}
\vspace{-0.4cm}
Several members from \href{http://geant4.cern.ch}{Geant4} collaboration
have participated the CRP for
\href{http://www-nds.iaea.org/charpar/charpar.htmlx}{Heavy Charged-Particle Interaction}:
\begin{itemize}
\item Collaborators from INFN-Catania 
(Geant4 hadrontherapy code by G.A.P. Cirrone, G.Cuttone, F. Di Rosa, F.Romano, G.Russo, M.Russo)
\item A.~Boudard and P.~Kaitaniemi, CEA-Saclay (INCL and ABLA models, Geant4 benchmarking platform) 
\item J.~M.~Quesada, Universidad de Sevilla) (issues related to precompound)
\item A.~Heikkinen, \href{http://www.hip.fi}{Helsinki Institute of physics} (coordination)
\end{itemize}

Also G.~Danielsen from Helsinki University of Technology is currently working with 
Geant4 benchmarking application that will implement agreed CRP tasks
(May -- August 2009. Work done in the framework Finnish CERN Summer Training 2009.)


\section{A platform for the IAEA benchmarking}
\vspace{-0.4cm}
We have prepared an open source Geant4 code, to allow non Geant4 members an access,
that acts as a  platform for benchmarking and other tasks to be made in the CRP.
\begin{itemize}
\item The code is based on Geant4 {\em Hadrontherapy} example prepared by INFN-Catania collaborators
and {\em test30} which is originally used for internal validation of Geant4 hadronic models.

\item \href{http://github.com/kaitanie/hadrontherapy/}{The code 
{\em radiotherapyData} is available in an open source repository}.
\end{itemize}

We have translated original INCL (courtesy of CEA-Saclay) and ABLA (GSI) codes into
Geant4 (first release December 2008 in Geant4 9.2) and 
have prepared physics list which will be validated to provide optimal Geant4 settings 
for therapeutic applications. 
\begin{itemize}
\item  A recent report describes INCL model: INDC(NDS)-0530 
(Joint ICTP-IAEA Advanced Workshop on Model Codes for Spallation Reactions).
\item We have also prepared a paper 
documenting physics results relevant in medical irradiations: 
{\em Simulation of light ion collisions from Intra Nuclear Cascade
(INCL-Fermi Breakup) relevant for medical irradiations and
radioprotection.}, AP/IE-08.
\attachfile{Acc_App_vienne2.pdf}
\end{itemize}

\section{Plans}
\vspace{-0.4cm}
\begin{itemize}
\item  When the Geant4 benchmarking software {\em Hadrontherapy} used in 
the agreed IAEA benchmarks is of production quality. (3~months)
\begin{itemize}
We are prepared to make the agreed proton and ions 
benchmarks for {\emph all} relevant Geant4 models. (12 months)

\end{itemize}
\item G.~Danielsen is preparing a thesis work 
{\em Simulating carbon beam fragmentation on water phantom with the Geant4 INCL/ABLA models}
\item Results of the Geant4 activity related to the CRP will be reported and published in 2010. (18 months)
\end{itemize}

\end{document}

